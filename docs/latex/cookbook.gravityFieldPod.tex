\section{Gravity field determination from precise orbit data (POD)}\label{cookbook.gravityFieldPod}

This cookbook chapter describes exemplarily the steps for determining the monthly gravity variations from precise orbit data (POD).

\subsection{Step 1: Preperation of data}

Following data have to be prepared monthly with an adequate sampling, e.g. 10 s using
\program{InstrumentConcatenate}:
\begin{itemize}
  \item Precise (kinematic) orbit data
  \item 3x3 covariance matrices data
  \item Initial orbit data used for precise orbit determination
  \item Star camera data
  \item Accelerometer data
\end{itemize}
Reduced sampling can be achieved by \program{InstrumentReduceSampling}. If the satellite mission does not provide any required
accelerometer data, these data can be generated via \program{SimulateAccelerometer}.

For satellite missions with less knowledge about the acting forces, it make sense to consider more than one state vector within an orbit revolution.
Otherwise the accuracy of the estimated parameters will decrease. This implies that shorter arcs are necessary. The assignment of the kinematic orbit
data as well as the 3x3 covariance matrices data to the arcs can be done with \program{InstrumentSynchronize}.

\subsection{Step 2: Conversion of the background gravity field}
\program{Gravityfield2SphericalHarmonicsVector} converts the static respectively background gravity field into spherical harmonics.

\subsection{Step 3: Preprocessing POD}
For determining the accuracies and weights of the kinematic orbits it is sufficient to make a least-square estimation with only certain parameters, due
to the fact that some parameters do not influence the estimation of the accuracies and weights.
This estimation is done with \program{PreprocessingPod}. Additional this program determines the temporal correlation of the kinematic orbit positions
x,y and z. If short arcs are used the setting \configClass{observation:podIntegral}{observationType:podIntegral} shall be used. This setting
considers the frictional forces by means of a macro model as well as the conservative and non-conservative forces.

\subsection{Step 4: Solving of normal equations system}
\program{NormalsSolverVCE} sets up the observation equations and summarized them to a normal equations system. The subsequent least-square estimation delivers
the parameters surcharges.

\subsection{Step 5: Determination of the estimated gravity field parameters}
The estimated parameters result from the re-addition of the background field, which is done in \program{MatrixCalculate}.

\subsection{Step 6: Conversion of the gravity field parameters}
\program{Gravityfield2PotentialCoefficients} converts the gravity field parameters into spherical harmonics.

% =============================================================
