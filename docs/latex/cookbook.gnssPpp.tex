\section{GNSS precise point positioning (PPP)}\label{cookbook.gnssPpp}

This cookbook chapter describes an example of precise point positioning (PPP) of a ground station using GPS and Galileo.
For information on how to generate the GNSS products (orbits, clocks, signal biases, etc.) required for PPP see the
cookbook chapter on \reference{GNSS satellite orbit determination and station network analysis}{cookbook.gnssNetwork}.

\subsection{Data and metadata preparation}\label{cookbook.gnssPpp:metadata}

PPP requires some data and metadata, most of which are
provided in GROOPS file formats at \url{https://ftp.tugraz.at/outgoing/ITSG/groops/data}
(marked with \textbf{*} below). These files are regularly updated. Alternatively, check the linked
file types and programs for information on how to create these files manually.

Satellite metadata:
\begin{itemize}
  \item \file{Transmitter info}{gnssStationInfo}\textbf{*}: PRN-SVN mapping, antenna offsets and orientations
  \item \file{Antenna definition}{gnssAntennaDefinition}\textbf{*}: antenna center variations
  \item \file{Receiver definition}{gnssReceiverDefinition}\textbf{*}: transmitted signal types (recommended)
  \item \file{Satellite macro model}{satelliteModel}\textbf{*}: box-wing model, mass, antenna thrust
\end{itemize}

Station metadata:
\begin{itemize}
  \item \file{Station info}{gnssStationInfo}\textbf{*}: antenna and receiver information, antenna reference point
        offsets, antenna orientations
  \item \file{Antenna definition}{gnssAntennaDefinition}\textbf{*}: antenna center offsets and variations
  \item \file{Receiver definition}{gnssReceiverDefinition}: observed signal types (optional)
  \item \file{Accuracy definition}{gnssAntennaDefinition}\textbf{*}: elevation and azimuth dependent accuracy
\end{itemize}

Most data files are not provided in GROOPS formats and have to be converted manually from external file formats.

Satellite data:
\begin{itemize}
  \item \textbf{Precise orbits}: precise orbits in CRF for orbit integration (see \program{Sp3Format2Orbit})
  \item \textbf{Precise clocks}: precise clocks (see \program{GnssClockRinex2InstrumentClock})
  \item \textbf{Attitude}: rotation from body frame to CRF (see \program{SimulateStarCameraGnss} or \program{GnssOrbex2StarCamera})
  \item \textbf{Signal biases}: code (and phase) biases (see \program{GnssSinexBias2SignalBias})
\end{itemize}

Station data:
\begin{itemize}
  \item \textbf{Observations}: receiver measurements (see \program{RinexObservation2GnssReceiver})
  \item \textbf{Troposphere model}\textbf{*}: mapping functions, i.e. VMF3 (see \program{ViennaMappingFunctionGrid2File},
        \program{ViennaMappingFunctionStation2File})
\end{itemize}

Additional data:
\begin{itemize}
  \item \textbf{Earth orientation parameters}\textbf{*}: IERS Earth rotation data
        (see \configClass{EarthRotation}{earthRotationType})
  \item \textbf{Gravity field}\textbf{*}: gravity field model for orbit preparation
        (see \configClass{Gravityfield}{gravityfieldType})
  \item \textbf{Tides and tidal deformation}\textbf{*}: models for astronomical, Earth, ocean, pole, ocean pole,
        and atmospheric tides (see \configClass{Tides}{tidesType})
\end{itemize}

\subsection{Preprocessing: Orbit preparation}\label{cookbook.gnssPpp:orbitPreparation}

Simple numerical interpolation of externally sourced precise orbits is not recommended, especially for low but commonly
used sampling rates of 15~minutes. Since \program{GnssProcessing} requires \file{variational equations}{variationalEquation}
as input anyway, a simple numerical integration and orbit fit is the recommended procedure to prepare
precise orbits.

The steps, which can be done for each satellite individually using \program{LoopPrograms}, are:
\begin{itemize}
  \item Create a \reference{global element}{general.configFiles:globalSection} \config{orbitSampling}=\verb|15*60|
        (or whatever the sampling of the orbits) of type \verb|double|.
  \item Resample the orbit to at least 1-minute sampling over the full 24~hours using \program{InstrumentResample}
        (polynomial interpolation with \config{polynomialDegree}=\verb|11|, \config{maxDataPointRange}=\verb|12*{orbitSampling}|, and
        \config{maxExtrapolationDistance}=\verb|{orbitSampling}| to extrapolate to the midnight epoch in case it is not included).
  \item In case attitude was not provided, use \program{SimulateStarCameraGnss} to simulate the attitude for the resampled orbit.
  \item Use \program{GnssPrn2SvnBlockVariables} to create a \reference{variable}{general.parser} containing the space vehicle number (SVN, e.g. G056).
  \item Integrate the orbit using \program{PreprocessingVariationalEquation}.
  \begin{itemize}
    \item Provide the \configFile{inputfileSatelliteModel}{satelliteModel} for the SVN, the resampled \configFile{inputfileOrbit}{instrument},
          and the simulated or provided \configFile{inputfileStarCamera}{instrument}.
    \item Add \configClass{gravityfield:potentialCoefficients}{gravityfieldType:potentialCoefficients} (e.g. GOCO06s static up to degree 20),
          \configClass{tides:astronomicalTide}{tidesType:astronomicalTide}, \configClass{tides:earthTide}{tidesType:earthTide}, and
          \configClass{miscAccelerations:relativisticEffect}{miscAccelerationsType:relativisticEffect} as \config{forces}.
    \item For \configClass{gradientField:potentialCoefficients}{gravityfieldType:potentialCoefficients} provide the same gravity field as above.
    \item Also provide \configClass{earthRotation}{earthRotationType} and \configClass{ephemerides}{ephemeridesType}.
  \end{itemize}
  \item Use \program{PreprocessingVariationalEquationOrbitFit} to fit the integrated orbit to the resampled reference orbit by estimating
        \configClass{parametrizationAcceleration:gnssSolarRadiation}{parametrizationAccelerationType:gnssSolarRadiation}
        (selecting at least the D0, D2, Y0, B1, B1 parameters). Set \config{iterationCount}=\verb|1|.
  \item Use \program{PreprocessingVariationalEquationOrbitFit} again, but this time instead of solar radiation parameters set up
        \configClass{stochasticPulse:uniformSampling}{timeSeriesType:uniformSampling} with
        \config{timeStart}=\verb|<mjd>+{orbitSampling}/86400|, \config{timeEnd}=\verb|<mjd>+1-{orbitSampling}/86400|,
        and \config{sampling}=\verb|{orbitSampling}/86400| (replace \verb|<mjd>| with MJD of the day to process).
\end{itemize}

The resulting \file{variational equations}{variationalEquation} files contain the integrated and fitted orbit, attitude, and Earth rotation
that are reqired by \program{GnssProcessing} for PPP.

\subsection{Processing of a ground station}\label{cookbook.gnssPpp:processing}

Once all metadata, data, and the orbits are prepared, the actual processing can begin.
Use \program{GnssProcessing} and define the \configClass{transmitter}{gnssParametrizationTransmitterType}
constellations, a \configClass{receiver:stationNetwork}{gnssParametrizationReceiverType:stationNetwork}, and
the list of \configClass{processingSteps}{gnssProcessingStepType}.

Since this program is very complex, an overview of the most important settings used for the example
processing in this cookbook chapter follows. Only config elements that deviate from the default values are listed.
For most input files the default value and annotation give a good hint of what is expected, so they are not
listed here.

The first step is setting the processing sampling, in this example it is 30~seconds.
The processing interval usually is a single 24-hour day,
\configClass{timeSeries:uniformSampling}{timeSeriesType:uniformSampling} with \config{timeStart}=\verb|<mjd>|,
\config{timeEnd}=\verb|<mjd>+1|, \config{sampling}=\verb|30/86400| (processing sampling).

Next define the GNSS constellations, in this case GPS and Galileo.
\begin{itemize}
  \item \configClass{transmitter:GPS}{gnssParametrizationTransmitterType:gps}
  \begin{itemize}
    \item \configFile{inputfileSignalBias}{gnssSignalBias}: code (and phase) biases
    \item \configFile{inputfileVariational}{variationalEquation}=\verb|preprocessing/variational.{prn}.dat| (prepared orbits)
    \item \config{estimateCodeBias}=\verb|no| (code biases should be provided)
    \item \config{estimatePhaseBias}=\verb|no| (code biases must be provided in case integer ambiguities are to be resolved)
  \end{itemize}
  \item \configClass{transmitter:Galileo}{gnssParametrizationTransmitterType:galileo} (use same settings as for GPS)
\end{itemize}

Define the station(s) to be processed. In this example it is a single stations, but you can also process multiple
stations at the same time.
Prepare the respective \file{station list files}{stringList} (one station name per line) beforehand.
\begin{itemize}
  \item \configClass{receiver:stationNetwork}{gnssParametrizationReceiverType:stationNetwork}
  \begin{itemize}
    \item \configFile{inputfileStationList}{stringList}: list of all stations to be processed
    \item \configFile{inputfileSignalBias}{gnssSignalBias}: leave empty, since we estimate our own biases
    \item \config{kinematicPositionEstimation}=\verb|yes| (since we want to estimate a position each epoch)
    \item \config{estimateCodeBias}=\verb|yes|
    \item \config{estimatePhaseBias}=\verb|yes|
    \item \configClass{troposphere:viennaMapping}{troposphereType} (e.g. VMF3 operational gridded)
    \item \configClass{troposphereWetEstimation:splines}{parametrizationTemporalType:splines} (e.g. degree 1 with 2-hourly
          nodes and daily intervals)
    \item \configClass{troposphereGradientEstimation:constant}{parametrizationTemporalType:constant}
    \item \configClass{troposphereGradientEstimation:trend}{parametrizationTemporalType:trend} (at center of day with
          \config{timeStep}=\verb|1|)
    \item \configClass{tides:earthTide}{tidesType:earthTide}: Earth tidal deformations (IERS conventions)
    \item \configClass{tides:doodsonHarmonicTide}{tidesType:doodsonHarmonicTide}: ocean tidal deformations
          (e.g. FES 2014b, \config{minDegree}=\verb|1|)
    \item \configClass{tides:doodsonHarmonicTide}{tidesType:doodsonHarmonicTide}: atmospheric tidal deformation
          (e.g. AOD1B RL06, \config{minDegree}=\verb|1|)
    \item \configClass{tides:poleTide}{tidesType:poleTide}: pole tidal deformations (IERS conventions)
    \item \configClass{tides:poleOceanTide}{tidesType:oceanPoleTide}: ocean pole tidal deformations (IERS conventions)
    \item \config{preprocessing}: settings for station-wise observation preprocessing
    \begin{itemize}
      \item We recommend to explicitly specify the signals to be processed and to make sure that at least transmitter
            code biases are provided for each of them, e.g. \configClass{useType}{gnssType}=\verb|C1CG|,
            \verb|C1CW|, \verb|C2WG|, \verb|L1*G|, \verb|L2*G|, ...).
      \item Set \config{codeMaxPositionDiff}=\verb|200| or larger in case your approximate position is not accurate or
            too many observations/epochs get disabled.
      \item Set \config{minEstimableEpochsRatio}=\verb|0.25| or lower in case too many observations/epochs get disabled
            and therefore the station is not processed at all.
    \end{itemize}
  \end{itemize}
\end{itemize}

In the next step, define the Earth rotation. The selected Earth rotation includes a high-frequency EOP model.
\begin{itemize}
  \item \configClass{earthRotation}{gnssParametrizationEarthRotationType}
  \begin{itemize}
    \item \configClass{earthRotation:file}{earthRotationType:file}
          with \configFile{inputfileEOP}{earthOrientationParameter}=\verb|timeSeries_EOP_IAU2000_desai.dat|
  \end{itemize}
\end{itemize}

Some parametrizations don't offer config elements for constraints on the respective parameters, so they have to be added
as general constraints. The parameters to be constrained can be selected by name (including support for wildcards).
The sigma for general constraints is always of the same unit as the parameter.
In this example only loose constraints on tropospheric parameters are required.
\begin{itemize}
  \item \configClass{constraints}{gnssParametrizationConstraintsType}: loose constraint troposphere estimates
  \begin{itemize}
    \item \configClass{parameters:wildcard}{parameterSelectorType:wildcard} with \config{object}=\verb|*|,
          \config{type}=\verb|troposphere*|, \config{temporal}=\verb|*|, \config{interval}=\verb|*|
    \item \config{sigma}=\verb|5| meters (wet)/degrees (gradient)
    \item \config{relativeToApriori}=\verb|yes|
  \end{itemize}
\end{itemize}

Finally, define the processing steps. This can be overwhelming at first, but offers a lot of flexibility.
The general steps are that an iterative least squares adjustment shall be performed for each station
(in this example there is only one) individually. Optionally, the integer ambiguities can be resolved as well
(only possible if transmitter phase biases are provided).
With all processing done, print the residual statistics and write the resulting output files.
\begin{itemize}
  \item \configClass{processingStep:forEachReceiverSeparately}{gnssProcessingStepType:forEachReceiverSeparately}:
        process all stations individually with fixed transmitter parameters
  \begin{itemize}
    \item \configClass{processingStep:selectParameters}{gnssProcessingStepType:selectParameters}: parameters
          for least squares adjustment (TEC biases cannot be estimated together with signal biases)
    \begin{itemize}
      \item \config{estimateReceiverTecBias}=\verb|no|
      \item \config{estimateTransmitterTecBias}=\verb|no|
      \item \config{constraintIonosphereSTEC}=\verb|no|
      \item \config{estimateIonosphereVTec}=\verb|no|
      \item \config{estimateIonosphereTecMap}=\verb|no|
    \end{itemize}
    \item \configClass{processingStep:estimate}{gnssProcessingStepType:estimate}: initial iterative estimation
          of all parameters (except TEC biases)
    \begin{itemize}
      \item \config{computeResiduals}=\verb|yes|
      \item \config{adjustSigma0}=\verb|groupedPhases|
      \item \config{computeWeights}=\verb|groupedPhases| (downweight code observations individually,
            but phase observations all grouped together)
      \item \config{convergenceThreshold}=\verb|0.005| meters
      \item \config{maxIterationCount}=\verb|6|
    \end{itemize}
    \item \configClass{processingStep:group}{gnssProcessingStepType:group}: estimate TEC biases in an intermediate step
    \begin{itemize}
      \item \configClass{processingStep:selectParameters}{gnssProcessingStepType:selectParameters}: set everything
            to \verb|no| except for the elements below (settings apply only to steps in current
            \configClass{group}{gnssProcessingStepType:group})
      \begin{itemize}
        \item \config{estimateReceiverTecBias}=\verb|yes|
        \item \config{constraintTransmitterSignalBiasZeroMean}=\verb|yes|
        \item \config{estimateTransmitterTecBias}=\verb|yes|
        \item \config{estimateIonosphereVTec}=\verb|yes|
        \item \config{estimateIonosphereTecMap}=\verb|yes|
        \item \config{constraintOther}=\verb|yes|
      \end{itemize}
      \item \configClass{processingStep:estimate}{gnssProcessingStepType:estimate}
      \begin{itemize}
        \item \config{computeResiduals}=\verb|yes|
        \item \config{convergenceThreshold}=\verb|0.005| meters
        \item \config{maxIterationCount}=\verb|3|
      \end{itemize}
    \end{itemize}
    \item \configClass{processingStep:estimate}{gnssProcessingStepType:estimate}: some more least squares iterations
          (same settings as first time except for \config{maxIterationCount}=\verb|2|)
    \item \configClass{processingStep:resolveAmbiguities}{gnssProcessingStepType:resolveAmbiguities}: (optional) integer
          ambiguity resolution (only possible if transmitter phase biases have been provided)
    \begin{itemize}
      \item \config{computeResiduals}=\verb|yes|
      \item \config{adjustSigma0}=\verb|groupedPhases|
      \item \config{computeWeights}=\verb|groupedPhases|
    \end{itemize}
    \item \configClass{processingStep:estimate}{gnssProcessingStepType:estimate}: (in case of ambiguity resolution) some more least squares iterations
          (same settings as first time except for \config{maxIterationCount}=\verb|2|)
  \end{itemize}
  \item \configClass{processingStep:printResidualStatistics}{gnssProcessingStepType:printResidualStatistics}
  \item \configClass{processingStep:writeResults}{gnssProcessingStepType:writeResults}
\end{itemize}

% =============================================================
