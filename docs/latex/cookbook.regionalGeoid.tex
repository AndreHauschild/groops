\section{Regional geoid determination}\label{cookbook.regionalGeoid}
This shows exemplary the computation of a regional geoid using terrestrial gravimetric observations
in combination with a global satellite model such as GOCO06s. The geoid is estimated in a least squares adjustment with
a parametrization using radial basis functions. A detailed desciption of the method is given in
Christian Pock (2017), Consistent Combination of Satellite and Terrestrial Gravity Field Observations in Regional Geoid Modeling.
Dissertation TU Graz.


\subsection{Gravimetric data}
Here it is assumed that the measured absolute gravity data is given at points in ellipsoidal coordinates.
The observed values should be converted to SI units~$m/s^2$.
\begin{itemize}
  \item \program{Matrix2GriddedData} to convert data from text file in tabular form.
\end{itemize}

\fig{!ht}{0.5}{regionalGeoidGravimetricData}{fig:regionalGeoidGravimetricData}{Distribution of gravimetric observations}

\subsection{Topography}
A high resolution topography model is needed to reduce the observations.
As the model heights are usually given in physical heights a reference geoid is needed to compute the correct ellipsoidal height.
\begin{itemize}
  \item \program{NetCdf2GridRectangular} convert into groops format.
  \item \program{Gravityfield2GriddedData}: Compute geoid heights using the GOCO06s model with
  \begin{itemize}
    \item \configClass{grid:file}{gridType:file} The topography grid
    \item \configClass{kernel:geoidHeight}{kernelType:geoidHeight}
    \item \configClass{gravityfield:potentialCoefficients}{gravityfieldType:potentialCoefficients} the GOCO06s model
    \item \configClass{gravityfield:potentialCoefficients}{gravityfieldType:potentialCoefficients}
    Subtract (\config{factor}=-1) GRS680 normal field.
  \end{itemize}
  \item \program{GriddedDataCalculate}: Generate a new combined griddedData file
        with the orthometric height (\verb|data0|) and the geoid height (\verb|data1|).
  \item \program{GriddedTopography2PotentialCoefficients}: Compute the gravitational potential in terms
        of spherical harmonics up to a maximum degree of the global satellite model.
        This is the part of the topography, which is already included in the global satellite model.
        The integration boundaries are \config{radialUpperBound}=\verb|data0+data1|
        and \config{radialLowerBound}=\verb|data1|.
\end{itemize}

\fig{!ht}{1.0}{regionalGeoidTopography}{fig:regionalGeoidTopography}{Topography and geoid heights}

\subsection{Reduce}
Calculate approximate reference gravity to reduce it from the observations.
\begin{itemize}
  \item \program{Gravityfield2AbsoluteGravity}
  \begin{itemize}
    \item \configClass{grid:File}{gridType:file} at observation positions
    \item \configClass{gravityfield:tides}{gravityfieldType:tides} Centrifugal potential
    \item \configClass{gravityfield:potentialCoefficients}{gravityfieldType:potentialCoefficients} full GOCO06s model
    \item \configClass{gravityfield:topography}{gravityfieldType:topography}
          (\config{radialUpperBound}=\verb|data0+data1|, \config{radialLowerBound}=\verb|data1|)
    \item \configClass{gravityfield:potentialCoefficients}{gravityfieldType:potentialCoefficients}
          Subtract (\config{factor}=-1) the potential part of the topography already included in the GOCO06s model.
  \end{itemize}
  \item \program{GriddedDataCalculate} to calculate observed minus computed.
  \item Large outliers can be removed in \program{GriddedDataCalculate} with \config{removalCriteria}.
\end{itemize}

\fig{!hb}{0.5}{regionalGeoidObserved-computed}{fig:regionalGeoidObserved-computed}{Gravity disturbances: observed minus computed}


\subsection{Radial Basis Functions (RBF)}
The residual gravity is parametrized in terms of Radial Basis Functions
\configClass{parametrizationGravity:radialBasis}{parametrizationGravityType:radialBasis}.
The basis functions should be distributed on a regular \configClass{grid}{gridType} covering
a somewhat larger area than the observations, see \configClass{border}{borderType}.
The shape of the functions \configClass{kernel:coefficients}{kernelType:coefficients} should reflect
the signal content of reduced observations and are defined by the coefficients.
\begin{itemize}
  \item \program{KernelDegreeVariances}
  \begin{itemize}
    \item \configClass{gravityfield:potentialCoefficients}{gravityfieldType:potentialCoefficients} accuracies of GOCO06s model
    \item \config{maxDegree}=7000. Complemented by Kaula's rule of thumb
  \end{itemize}
\end{itemize}
The maximum degree should correspond to the spatial resolution.
Rule of thumb: the number of spherical harmonic coefficients $(\text{maxDegree}+1)^2$ should roughly agree
to the number of grid points if they would cover the complete Earth.

\fig{!hb}{0.5}{regionalGeoidDegreeAmplitudes}{fig:regionalGeoidDegreeAmplitudes}{Degree amplitudes for the shape of the radial basis functions}

\subsection{Compute: Estimate parameters in a least squares adjustment}
Setup the observation equations and accumulate the system of normal equations.
\begin{itemize}
\item \program{NormalsBuild}
  \begin{itemize}
    \item \configClass{normalEquation:design}{normalEquationType:design} with
          \configClass{observation:terrestrial}{observationType:terrestrial}
    \begin{itemize}
      \item \configClass{kernel:disturbance}{kernelType:disturbance}
      \item \configClass{parametrizationGravity:radialBasis}{parametrizationGravityType:radialBasis}
            with \configClass{kernel:coefficients}{kernelType:coefficients}
    \end{itemize}
  \end{itemize}
\item \program{NormalsSolverVCE}
  \begin{itemize}
    \item \configClass{normalEquation:file}{normalEquationType:file} from \program{NormalsBuild}
    \item \configClass{normalEquation:regularization}{normalEquationType:regularization}
           towards zero means regularization towards the GOCO06s, which is reduced from the data.
  \end{itemize}
\end{itemize}

\subsection{Restore: Calculate the geoid solution}
Evaluate the estimated parameters and add back the reduced reference models.
\begin{itemize}
  \item \program{Gravityfield2GriddedData}: Compute approximate geoid heights using the GOCO06s model with
  \begin{itemize}
    \item \configClass{grid}{gridType} select a grid with target resolution at elliposid
    \item \configClass{kernel:geoidHeight}{kernelType:geoidHeight}
    \item \configClass{gravityfield:potentialCoefficients}{gravityfieldType:potentialCoefficients} GOCO06s model
    \item \configClass{gravityfield:potentialCoefficients}{gravityfieldType:potentialCoefficients}
    Subtract (\config{factor}=-1) GRS80 normal field.
  \end{itemize}
  \item \program{GriddedDataCalculate}: Move points from ellipsoid to geoid with \config{height}=\verb|data0|
  \item \program{Gravityfield2GriddedData}
  \begin{itemize}
    \item \configClass{grid:file}{gridType:file} from above.
    \item \configClass{kernel:geoidHeight}{kernelType:geoidHeight}
    \item \configClass{gravityfield:fromParametrization}{gravityfieldType:fromParametrization}
          The solution vector~$\M x$ together with the RBF parametrization
    \item \configClass{gravityfield:potentialCoefficients}{gravityfieldType:potentialCoefficients} GOCO06s model
    \item \configClass{gravityfield:topography}{gravityfieldType:topography}
          (\config{radialUpperBound}=\verb|data0+data1|, \config{radialLowerBound}=\verb|data1|)
    \item \configClass{gravityfield:potentialCoefficients}{gravityfieldType:potentialCoefficients}
          Subtract (\config{factor}=-1) the potential part of the topography already included in the GOCO06s model.
    \item \configClass{gravityfield:potentialCoefficients}{gravityfieldType:potentialCoefficients}
          Subtract (\config{factor}=-1) GRS80 normal field.
    \item \config{convertToHarmonics}=\verb|no|, otherwise the RBF are converted to harmonics up to degree 7000.
  \end{itemize}
  \item \program{GriddedDataCalculate}: Set \config{height}=0 of the computed geoid grid.
  \item \program{GridRectangular2NetCdf}
\end{itemize}

\fig{!hb}{0.6}{regionalGeoidGeoid}{fig:regionalGeoidGeoid}{Estimated geoid}


% =============================================================
