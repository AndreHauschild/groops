% auto generated by GROOPS
\subsection{Admittance}\label{general.fileFormat:admittance}
Interpolation matrix to create ocean minor tides from modeled major tides.
The file can be created with \program{DoodsonHarmonicsCalculateAdmittance} and used e.g. in
\configClass{doodsonHarmonicTide}{tidesType:doodsonHarmonicTide}.

See \program{DoodsonHarmonicsCalculateAdmittance}.


%==================================
\subsection{ArcList}\label{general.fileFormat:arcList}
With the \program{InstrumentSynchronize} an \file{instrument file}{instrument} can
be divided into time intervals and within the intervals into arcs.
This file provides the information about the mapping of arcs to time intervals.

This file can be used for the variational equation approach or \program{KalmanBuildNormals}.

\begin{verbatim}
groops arclist version=20200123
         32  # number of times
# time [MJD]               first arc
# ==================================
 58909.000000000000000000          0
 58910.000000000000000000          8
 58911.000000000000000000         17
 58912.000000000000000000         25
 58913.000000000000000000         29
 58914.000000000000000000         37
 58915.000000000000000000         45
 58916.000000000000000000         53
 58917.000000000000000000         61
 58918.000000000000000000         69
 58919.000000000000000000         78
 58920.000000000000000000         86
 58921.000000000000000000         95
 58922.000000000000000000        103
 58923.000000000000000000        112
 58924.000000000000000000        120
 58925.000000000000000000        128
 58926.000000000000000000        136
 58927.000000000000000000        144
 58928.000000000000000000        153
 58929.000000000000000000        161
 58930.000000000000000000        169
 58931.000000000000000000        177
 58932.000000000000000000        185
 58933.000000000000000000        193
 58934.000000000000000000        201
 58935.000000000000000000        210
 58936.000000000000000000        218
 58937.000000000000000000        226
 58938.000000000000000000        234
 58939.000000000000000000        242
 58940.000000000000000000        250
\end{verbatim}


%==================================
\subsection{DoodsonEarthOrientationParameter}\label{general.fileFormat:doodsonEarthOrientationParameter}
Corrections for Earth orientation parameters (EOP) ($x_p, y_p, UT1, LOD$)
as cos/sin oscillations for a list of doodson tidal frequencies.

\begin{verbatim}
groops doodsonEarthOrientationParameter version=20200123
        11 # number of constituents
# dood.   xpCos [arcsec]            xpSin [arcsec]            ypCos [arcsec]            ypSin [arcsec]            ut1Cos [sec]              ut1Sin [sec]              lodCos [sec]              lodSin [sec]             name
# ===========================================================================================================================================================================================================================
 155.645  2.399999999999999786e-07  1.799999999999999971e-07  1.799999999999999971e-07 -2.399999999999999786e-07 -2.000000000000000042e-08  2.000000000000000042e-08 -1.499999999999999932e-07 -1.400000000000000095e-07 ""
 155.655 -8.220000000000000920e-06 -6.280000000000000012e-06 -6.280000000000000012e-06  8.220000000000000920e-06  7.899999999999999537e-07 -8.599999999999999187e-07  5.219999999999999150e-06  4.829999999999999496e-06 m1
 155.665 -1.649999999999999872e-06 -1.260000000000000007e-06 -1.260000000000000007e-06  1.649999999999999872e-06  1.600000000000000033e-07 -1.700000000000000135e-07  1.049999999999999900e-06  9.700000000000000302e-07 ""
 157.455 -1.539999999999999867e-06 -1.199999999999999946e-06 -1.199999999999999946e-06  1.539999999999999867e-06  1.499999999999999932e-07 -1.600000000000000033e-07  9.799999999999999345e-07  8.899999999999999491e-07 chi1
 157.465 -3.400000000000000270e-07 -2.599999999999999988e-07 -2.599999999999999988e-07  3.400000000000000270e-07  2.999999999999999732e-08 -4.000000000000000084e-08  2.099999999999999746e-07  1.999999999999999909e-07 ""
 161.557  9.999999999999999547e-08  8.000000000000000167e-08  8.000000000000000167e-08 -9.999999999999999547e-08 -1.000000000000000021e-08  1.000000000000000021e-08 -7.000000000000000477e-08 -4.999999999999999774e-08 ""
 162.556  2.549999999999999726e-06  2.019999999999999718e-06  2.019999999999999718e-06 -2.549999999999999726e-06 -2.099999999999999746e-07  2.899999999999999763e-07 -1.799999999999999919e-06 -1.289999999999999931e-06 pi1
 163.545 -4.899999999999999672e-07 -3.799999999999999616e-07 -3.799999999999999616e-07  4.899999999999999672e-07  4.000000000000000084e-08 -5.999999999999999464e-08  3.499999999999999842e-07  2.399999999999999786e-07 ""
 163.555  4.272999999999999238e-05  3.010999999999999801e-05  3.010999999999999801e-05 -4.272999999999999238e-05 -3.079999999999999734e-06  5.219999999999999150e-06 -3.270999999999999684e-05 -1.929999999999999817e-05 p1
 164.554 -3.599999999999999943e-07 -2.800000000000000191e-07 -2.800000000000000191e-07  3.599999999999999943e-07  2.999999999999999732e-08 -4.000000000000000084e-08  2.700000000000000090e-07  1.700000000000000135e-07 ""
 164.556 -1.029999999999999879e-06 -7.999999999999999638e-07 -7.999999999999999638e-07  1.029999999999999879e-06  8.000000000000000167e-08 -1.199999999999999893e-07  7.599999999999999233e-07  4.899999999999999672e-07 ""
\end{verbatim}


%==================================
\subsection{DoodsonHarmonic}\label{general.fileFormat:doodsonHarmonic}
Ocean tides are represented as time variable gravitational potential
and is given by a fourier expansion
\begin{equation}
V(\M x,t) = \sum_{f} V_f^c(\M x)\cos(\Theta_f(t)) + V_f^s(\M x)\sin(\Theta_f(t)),
\end{equation}
where $V_f^c(\M x)$ and $V_f^s(\M x)$ are spherical harmonics.
The $\Theta_f(t)$ are the arguments of the tide constituents $f$:
\begin{equation}
\Theta_f(t) = \sum_{i=1}^6 n_f^i\beta_i(t),
\end{equation}
where $\beta_i(t)$ are the Doodson's fundamental arguments ($\tau,s,h,p,N',p_s$) and $n_f^i$
are the Doodson multipliers for the term at frequency~$f$.

To extract the potential coefficients of $V_f^c$ and $V_f^s$
for each frequency $f$ use \program{DoodsonHarmonics2PotentialCoefficients}.

See also \program{PotentialCoefficients2DoodsonHarmonics}.


%==================================
\subsection{EarthOrientationParameter}\label{general.fileFormat:earthOrientationParameter}
Earth Orientation Parameter (EOP) as provided by the International Earth Rotation and Reference Systems Service (IERS) (e.g \verb|EOP 14 C04 (IAU2000A)|).

See \program{IersC04IAU2000EarthOrientationParameter}, \program{IersRapidIAU2000EarthOrientationParameter}.

\begin{verbatim}
groops earthOrientationParameter version=20200123
       9641 # number of epochs
# UTC [MJD]                 xp [arcsec]               yp [arcsec]               deltUT [sec]              LOD [sec]                 dX [arcsec]               dY [arcsec]
# ====================================================================================================================================================================================
  5.894700000000000000e+04  5.690599999999999825e-02  4.099130000000000273e-01 -2.316246000000000138e-01  1.636400000000000094e-03 -2.900000000000000017e-05  5.800000000000000034e-05
  5.894800000000000000e+04  5.771400000000000141e-02  4.110159999999999925e-01 -2.332083000000000073e-01  1.520099999999999923e-03 -6.000000000000000152e-05  2.199999999999999943e-05
  5.894900000000000000e+04  5.813000000000000111e-02  4.120099999999999874e-01 -2.346157000000000104e-01  1.293099999999999935e-03 -7.200000000000000182e-05  3.199999999999999855e-05
  5.895000000000000000e+04  5.854100000000000276e-02  4.129849999999999910e-01 -2.357567999999999886e-01  9.872999999999999832e-04 -7.600000000000000418e-05  5.899999999999999754e-05
  5.895100000000000000e+04  5.908599999999999963e-02  4.139869999999999939e-01 -2.366149999999999920e-01  7.075000000000000126e-04 -8.000000000000000654e-05  8.600000000000000331e-05
  5.895200000000000000e+04  5.976900000000000268e-02  4.154180000000000095e-01 -2.372105999999999937e-01  4.798000000000000073e-04 -8.399999999999999535e-05  1.129999999999999955e-04
  5.895300000000000000e+04  6.095400000000000124e-02  4.167310000000000181e-01 -2.375994999999999913e-01  3.118999999999999919e-04 -8.700000000000000051e-05  1.399999999999999877e-04
  5.895400000000000000e+04  6.210199999999999748e-02  4.180929999999999924e-01 -2.378588000000000091e-01  1.710000000000000094e-04 -9.100000000000000287e-05  1.669999999999999935e-04
  5.895500000000000000e+04  6.290999999999999370e-02  4.196619999999999795e-01 -2.380454999999999932e-01  1.719000000000000042e-04 -6.000000000000000152e-06  8.100000000000000375e-05
  5.895600000000000000e+04  6.385599999999999610e-02  4.214060000000000028e-01 -2.382557999999999898e-01  2.683000000000000163e-04  1.029999999999999964e-04 -3.600000000000000091e-05
  5.895700000000000000e+04  6.455500000000000127e-02  4.229890000000000039e-01 -2.385857000000000117e-01  4.040000000000000080e-04  1.019999999999999992e-04 -2.000000000000000164e-05
  5.895800000000000000e+04  6.440300000000000191e-02  4.242549999999999932e-01 -2.390210000000000112e-01  4.910999999999999567e-04  5.899999999999999754e-05  4.399999999999999886e-05
\end{verbatim}


%==================================
\subsection{EarthTide}\label{general.fileFormat:earthTide}
Containing the Love numbers together with frequency corrections to compute
the gravitational potential and the geometric displacements due to solid Earth tides.
It is used by \configClass{tides}{tidesType}.


%==================================
\subsection{Ephemerides}\label{general.fileFormat:ephemerides}
Ephemerides of sun, moon, and planets stored as coefficients of Chebyshev polynomials.
Used in \configClass{Ephemerides:jpl}{ephemeridesType:jpl}.

See also \program{JplAscii2Ephemerides}.


%==================================
\subsection{GnssAntennaDefinition}\label{general.fileFormat:gnssAntennaDefinition}
Contains a list of GNSS antennas which are identified by its
name (type), serial, and radome. Each antenna consists of
antenna center offsets (ACO) and antenna center variations (ACV)
for different signal \configClass{types}{gnssType} (code and phase).
The ACV values for each type are stored in an elevation and azimuth dependent grid.

%New \config{antenna} with \config{pattern}s for code (\config{type}=\verb|C**|) and phase
%(\config{type}=\verb|L**|).
%The standard deviation is expressed e.g. with \config{values}=\verb|0.001/cos(2*PI/180*zenith)|.

See also \program{GnssAntennaDefinitionCreate}, \program{GnssAntex2AntennaDefinition}.

\fig{!hb}{1.0}{fileFormatGnssAntennaDefinition}{fig:fileFormatGnssAntennaDefinition}{Antenna center variations of ASH701945D\_M for two frequencies of GPS and GLONASS}

\begin{verbatim}
<?xml version="1.0" encoding="UTF-8"?>
<groops type="antennaDefinition" version="20190429">
    <antennaCount>65</antennaCount>
    ...
    <antenna>
        <name>BLOCK IIIA</name>
        <serial>G074</serial>
        <radome>2018-109A</radome>
        <comment>PCO provided by the Aerospace Corporation, PV from estimations by ESA/CODE</comment>
        <pattern>
            <count>3</count>
            <cell>
                <type>*1*G**</type>
                <offset>
                    <x>-1.23333333333333e-03</x>
                    <y>4.33333333333333e-04</y>
                    <z>3.15200000000000e-01</z>
                </offset>
                <dZenit>1.00000000000000e+00</dZenit>
                <pattern>
                    <type>0</type>
                    <rows>1</rows>
                    <columns>18</columns>
                    <cell row="0" col="0">1.39000000000000e-02</cell>
                    <cell row="0" col="1">1.28000000000000e-02</cell>
                    <cell row="0" col="2">1.02000000000000e-02</cell>
                    <cell row="0" col="3">5.80000000000000e-03</cell>
                    <cell row="0" col="4">1.10000000000000e-03</cell>
                    <cell row="0" col="5">-4.50000000000000e-03</cell>
                    <cell row="0" col="6">-9.70000000000000e-03</cell>
                    <cell row="0" col="7">-1.28000000000000e-02</cell>
                    <cell row="0" col="8">-1.34000000000000e-02</cell>
                    <cell row="0" col="9">-1.18000000000000e-02</cell>
                    <cell row="0" col="10">-8.90000000000000e-03</cell>
                    <cell row="0" col="11">-4.50000000000000e-03</cell>
                    <cell row="0" col="12">1.20000000000000e-03</cell>
                    <cell row="0" col="13">7.20000000000000e-03</cell>
                    <cell row="0" col="14">1.33000000000000e-02</cell>
                    <cell row="0" col="15">1.33000000000000e-02</cell>
                    <cell row="0" col="16">1.33000000000000e-02</cell>
                    <cell row="0" col="17">1.33000000000000e-02</cell>
                </pattern>
            </cell>
            ...
        </pattern>
    </antenna>
</groops>
\end{verbatim}


%==================================
\subsection{GnssReceiverDefinition}\label{general.fileFormat:gnssReceiverDefinition}
Contains a list of GNSS receivers which are identified by its
name, serial, and version. Defines for each receiver a list of
signal \configClass{types}{gnssType} which can be observed.
Can also be used for GNSS transmitters to define a list of
transmitted signal types. For GLONASS satellites the frequency
number can be stored in the \emph{version} field.

See \program{GnssReceiverDefinitionCreate}.

\begin{verbatim}
<?xml version="1.0" encoding="UTF-8"?>
<groops type="receiverDefinition" version="20190429">
    <receiverCount>112</receiverCount>
    <receiver>
        <name>GLONASS</name>
        <serial>R779</serial>
        <version>2</version>
        <comment/>
        <types>
            <count>4</count>
            <cell>*1CR**J</cell>
            <cell>*1PR**J</cell>
            <cell>*2CR**J</cell>
            <cell>*2PR**J</cell>
        </types>
    </receiver>
    ...
    <receiver>
        <name>GLONASS-K1</name>
        <serial>R802</serial>
        <version>7</version>
        <comment/>
        <types>
            <count>10</count>
            <cell>*1CR**O</cell>
            <cell>*1PR**O</cell>
            <cell>*2CR**O</cell>
            <cell>*2PR**O</cell>
            <cell>*3IR**</cell>
            <cell>*3QR**</cell>
            <cell>*4AR**</cell>
            <cell>*4BR**</cell>
            <cell>*6AR**</cell>
            <cell>*6BR**</cell>
        </types>
    </receiver>
</groops>
\end{verbatim}


%==================================
\subsection{GnssSignalBias}\label{general.fileFormat:gnssSignalBias}
Signal biases of GNSS transmitters or receivers for different \configClass{gnssType}{gnssType}.

\begin{verbatim}
groops gnssSignalBias version=20200123
          5 # number of signals
# type   bias [m]
# ===============================
 C1CG06 -1.752461109688110974e-01
 C1WG06  4.005884595055994590e-02
 C2WG06  6.597469378913034532e-02
 L1*G06 -2.736169875580296909e-02
 L2*G06  3.422596762686257871e-02
 \end{verbatim}

See also \program{GnssProcessing}, \program{GnssSimulateReceiver}, \program{GnssSignalBias2Matrix}, \program{GnssSignalBias2SinexBias}.


%==================================
\subsection{GriddedData}\label{general.fileFormat:griddedData}
List of arbitrarily distributed points defined by geographic coordinates and ellipsoidal
height. Each point can also have an associated area
(projected on the unit sphere with a total area of $4\pi$).
This file format supports multiple values per point (called \verb|data0|, \verb|data1| and so on).

For regular gridded data and binary format (\verb|*.dat|) a more efficient storage scheme is used.

See also: \program{GriddedDataCreate}.

\begin{verbatim}
groops griddedData version=20200123
 1  2  6.378137000000000000e+06  6.356752314140356146e+06 72 # hasArea, data columns, ellipoid a, ellipoid b, data rows
# longitude [deg]           latitude [deg]            height [m]                unit areas [-]             data0                     data1
# ===========================================================================================================================================================
 -1.650000000000000000e+02  7.500000000000000000e+01  0.000000000000000000e+00  7.014893453974438420e-02  1.000000000000000000e+00  2.000000000000000000e+00
 -1.350000000000000000e+02  7.500000000000000000e+01  0.000000000000000000e+00  7.014893453974438420e-02  1.000000000000000000e+00  2.000000000000000000e+00
 -1.050000000000000142e+02  7.500000000000000000e+01  0.000000000000000000e+00  7.014893453974438420e-02  1.000000000000000000e+00  2.000000000000000000e+00
 -7.500000000000001421e+01  7.500000000000000000e+01  0.000000000000000000e+00  7.014893453974438420e-02  1.000000000000000000e+00  2.000000000000000000e+00
 -4.500000000000002132e+01  7.500000000000000000e+01  0.000000000000000000e+00  7.014893453974438420e-02  1.000000000000000000e+00  2.000000000000000000e+00
 -1.500000000000002132e+01  7.500000000000000000e+01  0.000000000000000000e+00  7.014893453974438420e-02  1.000000000000000000e+00  2.000000000000000000e+00
  1.499999999999997691e+01  7.500000000000000000e+01  0.000000000000000000e+00  7.014893453974438420e-02  1.000000000000000000e+00  2.000000000000000000e+00
  4.499999999999997868e+01  7.500000000000000000e+01  0.000000000000000000e+00  7.014893453974438420e-02  1.000000000000000000e+00  2.000000000000000000e+00
  7.499999999999997158e+01  7.500000000000000000e+01  0.000000000000000000e+00  7.014893453974438420e-02  1.000000000000000000e+00  2.000000000000000000e+00
  1.049999999999999574e+02  7.500000000000000000e+01  0.000000000000000000e+00  7.014893453974438420e-02  1.000000000000000000e+00  2.000000000000000000e+00
  1.349999999999999432e+02  7.500000000000000000e+01  0.000000000000000000e+00  7.014893453974438420e-02  1.000000000000000000e+00  2.000000000000000000e+00
  1.649999999999999432e+02  7.500000000000000000e+01  0.000000000000000000e+00  7.014893453974438420e-02  1.000000000000000000e+00  2.000000000000000000e+00
 -1.650000000000000000e+02  4.500000000000000711e+01  0.000000000000000000e+00  1.916504532594049681e-01  1.000000000000000000e+00  2.000000000000000000e+00
 -1.350000000000000000e+02  4.500000000000000711e+01  0.000000000000000000e+00  1.916504532594049681e-01  1.000000000000000000e+00  2.000000000000000000e+00
\end{verbatim}



%==================================
\subsection{GriddedDataTimeSeries}\label{general.fileFormat:griddedDataTimeSeries}
Time series of data for arbitrarily distributed points defined by geographic coordinates and ellipsoidal
height. The data can be temporal interpolated by \reference{basis splines}{fundamentals.basisSplines}.
The file format consists of a \file{griddedData}{griddedData}, a time series, and
for each spatial point and spline node pair multiple values called \verb|data0|, \verb|data1|, \ldots.

A GriddedDataTimeSeries can be generated from individual \file{griddedData}{griddedData} with the program
\program{GriddedData2GriddedDataTimeSeries}. Vice-versa, a GriddedDataTimeSeries can be evaluated at a
specific time stamp to obtain a \file{griddedData}{griddedData} with \program{GriddedDataTimeSeries2GriddedData}.


%==================================
\subsection{Instrument}\label{general.fileFormat:instrument}
This template file format can store different observations in a epoch wise manner. Each epoch consists of a time and
additional data, e.g orbits, accelerometer data, star camera quaternions (see \configClass{InstrumentType}{instrumentTypeType}).
The time series can be divided in several arcs (see \program{InstrumentSynchronize}).

Also a \file{matrix}{matrix} file is allowed as one single arc. The first column must contain times [MJD]. Without any extra column
the instrument type is INSTRUMENTTIME, with one additional column the type is MISCVALUE, and for more columns the type
MISCVALUES is used.

\begin{verbatim}
groops instrument version=20200123
# SATELLITETRACKING
         -9         60  # instrument type, number of arcs
# Time [MJD]               data0: range [m]          data1: range-rate [m/s]   data2: range-acc [m/s^2]
# =====================================================================================================
         12 # number of epochs of 1. arc
 54588.000000000000000000 -5.074649470097549492e+05  5.755440207134928654e-01  1.877605261528093308e-03
 54588.000057870370255841 -5.074620458130163024e+05  5.849357691551860805e-01  1.878948916234051596e-03
 54588.000115740740966430 -5.074590976427756250e+05  5.943331739937073310e-01  1.879937220634776869e-03
 54588.000173611111222272 -5.074561024756557308e+05  6.037340169611068452e-01  1.880370529387525701e-03
 54588.000231481481478113 -5.074530602992626373e+05  6.131368121270999172e-01  1.880680632122925426e-03
 54588.000289351851733954 -5.074499711071007187e+05  6.225398878861636565e-01  1.880495369480403561e-03
 54588.000347222222444543 -5.074468349029610981e+05  6.319414138081351773e-01  1.880073731783055927e-03
 54588.000405092592700385 -5.074436516971451929e+05  6.413404243585696385e-01  1.879464843086203459e-03
 54588.000462962962956226 -5.074404215058300761e+05  6.507353310092597320e-01  1.878578987216372124e-03
 54588.000520833333212067 -5.074371443491023383e+05  6.601267978060636477e-01  1.877878184949659246e-03
 54588.000578703703922656 -5.074338202460713219e+05  6.695136489207137442e-01  1.876962042758626532e-03
 54588.000636574074178498 -5.074304492190054734e+05  6.788964444122400632e-01  1.876091925462087043e-03
         12 # number of epochs of 2. arc
 54588.000694444444434339 -5.074270312892858055e+05  6.882748400534359767e-01  1.875376456928801432e-03
 54588.000752314814690180 -5.074235664742725785e+05  6.976508178537534910e-01  1.874929898412159559e-03
 54588.000810185185400769 -5.074200547868391732e+05  7.070236200716006891e-01  1.874312324351668077e-03
 54588.000868055555656611 -5.074164962409950094e+05  7.163943828291452487e-01  1.873924188388115340e-03
 54588.000925925925912452 -5.074128908454515622e+05  7.257639682023964145e-01  1.874025826380292404e-03
 54588.000983796296168293 -5.074092386012640782e+05  7.351333608427884636e-01  1.873680487441316657e-03
 54588.001041666666878882 -5.074055395130896359e+05  7.445020815182646912e-01  1.873849502509668122e-03
 54588.001099537037134724 -5.074017935789784533e+05  7.538716732272922050e-01  1.873971633320137753e-03
 54588.001157407407390565 -5.073980007962241652e+05  7.632414098560330595e-01  1.873984767500571974e-03
 54588.001215277777646406 -5.073941611626467784e+05  7.726123093411200182e-01  1.874295246964456478e-03
 54588.001273148148356995 -5.073902746728868224e+05  7.819835205798950639e-01  1.874226146744964808e-03
 54588.001331018518612836 -5.073863413272026228e+05  7.913547196412918927e-01  1.874173804634685515e-03
\end{verbatim}


%==================================
\subsection{Matrix}\label{general.fileFormat:matrix}
Stores matrices and vectors. Only one triangle is written for symmetric or triangular matrices.

The header (the matrix definition) is optional.
Therefore a pure text with only numbers in columns are also allowed.
This simplifies the handling of external data.

Instead of a matrix file also an \file{instrument}{instrument} file is allowed.
The first column is the time [MJD], the other columns depends on the instrument type.

\begin{verbatim}
groops matrix version=20200123
LowerSymmetricMatrix( 4 x 4 )
  1.000000000000000000e+00
  0.000000000000000000e+00  1.000000000000000000e+00
  0.000000000000000000e+00  0.000000000000000000e+00  1.000000000000000000e+00
  0.000000000000000000e+00  0.000000000000000000e+00  0.000000000000000000e+00  1.000000000000000000e+00
\end{verbatim}



%==================================
\subsection{MeanPolarMotion}\label{general.fileFormat:meanPolarMotion}
The mean pole of the Earth rotation is represented by a polynomial in a time interval.

\begin{verbatim}
<?xml version="1.0" encoding="UTF-8"?>
<groops type="meanPolarMotion" version="20200123">
  <meanPolarMotion>
    <intervalCount>2</intervalCount>
    <interval>
      <timeStart>42778.0000000000000000</timeStart>
      <degree>3</degree>
      <xp>5.59741000000000e-02</xp>
      <xp>1.82430000000000e-03</xp>
      <xp>1.84130000000000e-04</xp>
      <xp>7.02400000000000e-06</xp>
      <yp>3.46346000000000e-01</yp>
      <yp>1.78960000000000e-03</yp>
      <yp>-1.07290000000000e-04</yp>
      <yp>-9.08000000000000e-07</yp>
    </interval>
    <interval>
      <timeStart>55197.0000000000000000</timeStart>
      <degree>1</degree>
      <xp>2.35130000000000e-02</xp>
      <xp>7.61410000000000e-03</xp>
      <yp>3.58891000000000e-01</yp>
      <yp>-6.28700000000000e-04</yp>
    </interval>
  </meanPolarMotion>
</groops>
\end{verbatim}


%==================================
\subsection{NormalEquation}\label{general.fileFormat:normalEquation}
Stores a \reference{system of normal equations}{normalEquationType}
\begin{equation}
  \M N \hat{\M x} = \M n.
\end{equation}.
This file format consists of multiple files.
The file name \verb|normals.dat.gz| corresponds to the following files:
\begin{itemize}
\item \verb|normals.dat.gz| or \verb|normals.00.00.dat.gz| ... \verb|normals.0n.0n.dat.gz|:
      the normal matrix $\M N$ as \file{matrix}{matrix},
\item \verb|normals.rightHandSide.dat.gz|:
      the right hand side(s) $\M n$ as \file{matrix}{matrix},
\item \verb|normals.parameterNames.txt|: \file{parameter names}{parameterName},
\item \verb|normals.info.xml|:
\     u.a. containing the number of observations and the quadratic sum of (reduced) observations $\M l^T\M P\M l$.
\end{itemize}
A large normal matrix may be splitted into blocks and stored in multiple files.
The block row/column number is indicated in the file name.
Only the upper blocks of the sysmmetric matrix are considered.
Matrix in blocks can be distributed on muliple nodes in parallel mode to efficiently use distributed memory.


%==================================
\subsection{OceanPoleTide}\label{general.fileFormat:oceanPoleTide}
Describes the reaction of the ocean mass to the change
of the centrifugal potential (polar wobble) in terms spherical harmonics.

See also \program{Iers2OceanPoleTide}.


%==================================
\subsection{ParameterName}\label{general.fileFormat:parameterName}
Name of parameters of a system of \file{normal equations}{normalEquation} or \file{solution vector}{matrix}.

A parameter name is a string \verb|<object>:<type>:<temporal>:<interval>| containg four parts divided by \verb|:|
\begin{enumerate}
\item object: Object this parameter refers to, e.g. \verb|graceA|, \verb|G023|, \verb|earth|, \ldots
\item type: Type of this parameter, e.g. \verb|accBias|, \verb|position.x|, \ldots
\item temporal: Temporal representation of this parameter, e.g. \verb|trend|, \verb|polynomial.degree1|, \ldots
\item interval: Interval/epoch this parameter represents, e.g. \verb|2017-01-01_00-00-00_2017-01-02_00-00-00|, \verb|2018-01-01_00-00-00|.
\end{enumerate}
In the documentation a star (\verb|*|) in the name means this part is untouched and useally set by other classes.
Times are written as \verb|yyyy-mm-dd_hh-mm-ss| and intervals (if not empty) as \verb|<timeStart>_<timeEnd>|.

See \program{ParameterNamesCreate}.

\begin{verbatim}
groops parameterName version=20200123
# object:type:temporal:interval
# =============================
      10080 # number of parameters
 karr:position.x::2018-06-01_00-00-00_2018-06-02_00-00-00
 karr:position.y::2018-06-01_00-00-00_2018-06-02_00-00-00
 karr:position.z::2018-06-01_00-00-00_2018-06-02_00-00-00
 karr:troposphereWet:spline.n1:2018-06-01_00-00-00_2018-06-01_02-00-00
 karr:troposphereWet:spline.n1:2018-06-01_00-00-00_2018-06-01_04-00-00
 karr:troposphereWet:spline.n1:2018-06-01_02-00-00_2018-06-01_06-00-00
 karr:troposphereWet:spline.n1:2018-06-01_04-00-00_2018-06-01_08-00-00
 karr:troposphereWet:spline.n1:2018-06-01_06-00-00_2018-06-01_10-00-00
 karr:troposphereWet:spline.n1:2018-06-01_08-00-00_2018-06-01_12-00-00
 karr:troposphereWet:spline.n1:2018-06-01_10-00-00_2018-06-01_14-00-00
 karr:troposphereWet:spline.n1:2018-06-01_12-00-00_2018-06-01_16-00-00
 karr:troposphereWet:spline.n1:2018-06-01_14-00-00_2018-06-01_18-00-00
 karr:troposphereWet:spline.n1:2018-06-01_16-00-00_2018-06-01_20-00-00
 karr:troposphereWet:spline.n1:2018-06-01_18-00-00_2018-06-01_22-00-00
 karr:troposphereWet:spline.n1:2018-06-01_20-00-00_2018-06-02_00-00-00
 karr:troposphereWet:spline.n1:2018-06-01_22-00-00_2018-06-02_00-00-00
 karr:troposphereGradient.x:spline.n1:2018-06-01_00-00-00_2018-06-02_00-00-00
 karr:troposphereGradient.y:spline.n1:2018-06-01_00-00-00_2018-06-02_00-00-00
 karr:troposphereGradient.x:spline.n1:2018-06-01_00-00-00_2018-06-02_00-00-00
 karr:troposphereGradient.y:spline.n1:2018-06-01_00-00-00_2018-06-02_00-00-00
 karr:signalBias01(+1.00L1CG**)::
 karr:signalBias02(+1.00L2WG**)::
 karr:signalBias03(+1.00L2XG**)::
 G01:solarRadiationPressure.ECOM.D0::
 G01:solarRadiationPressure.ECOM.DC2::
 G01:solarRadiationPressure.ECOM.DS2::
 G01:solarRadiationPressure.ECOM.Y0::
 G01:solarRadiationPressure.ECOM.B0::
 G01:solarRadiationPressure.ECOM.BC1::
 G01:solarRadiationPressure.ECOM.BS1::
 G01:stochasticPulse.x::2018-06-01_12-00-00
 G01:stochasticPulse.y::2018-06-01_12-00-00
 G01:stochasticPulse.z::2018-06-01_12-00-00
 G01:arc0.position0.x::
 G01:arc0.position0.y::
 G01:arc0.position0.z::
 G01:arc0.velocity0.x::
 G01:arc0.velocity0.y::
 G01:arc0.velocity0.z::
 G01:signalBias01(-1.00C1CG01)::
 G01:signalBias02(+1.00L1*G01)::
 G01:signalBias03(+1.00L2*G01)::
\end{verbatim}


%==================================
\subsection{Platform}\label{general.fileFormat:platform}
Defines a platform with a local coordinate frame equipped with instruments.
The platform might be a reference station, a low Earth satellite,
or a transmitting GNSS satellite and is referenced by a marker name and number.
The reference point (marker or center of mass (CoM)) can change in time
relative to the local frame.

Each equipped instrument is described at least by the following information
\begin{itemize}
\item name
\item serial number
\item coordinates in the local frame
\item a time interval in which the instrument was active
\item the orientation for antennas and reflectors.
\end{itemize}

For GNSS satellites the platform defines the PRN. The different assigned SVNs
are defined by the transmitting antennas.

Platforms for GNSS stations can be created from station log files with
\program{GnssStationLog2Platform}. Platforms for GNSS satellites
can be created from an ANTEX file with \program{GnssAntex2AntennaDefinition}.

See also \program{PlatformCreate}.

\fig{!hb}{0.8}{fileFormatPlatform}{fig:fileFormatPlatform}{Platform for stations, LEOs, and GNSS satellites.}


%==================================
\subsection{Polygon}\label{general.fileFormat:polygon}
List of longitude and latitudes to describe borders, e.g. river basins or continents.
It is used in \configClass{border:polygon}{borderType:polygon}.

\begin{verbatim}
groops polygon version=20200123
          2  # number of polygons
          6  # number of points (1. polygon)
# longitude [deg]           latitude [deg]
# ==================================================
 -1.598200000000000216e+02  2.203000000000000114e+01
 -1.596200000000000045e+02  2.189999999999999858e+01
 -1.593799999999999955e+02  2.189999999999999858e+01
 -1.593000000000000114e+02  2.221999999999999886e+01
 -1.595800000000000125e+02  2.221999999999999886e+01
 -1.598200000000000216e+02  2.203000000000000114e+01
          5  # number of points (2. polygon)
# longitude [deg]           latitude [deg]
# ==================================================
 -7.900000000000000000e+01  2.669999999999999929e+01
 -7.870000000000000284e+01  2.650000000000000000e+01
 -7.823000000000000398e+01  2.667000000000000171e+01
 -7.793000000000000682e+01  2.667000000000000171e+01
 -7.779999999999999716e+01  2.646999999999999886e+01
\end{verbatim}



%==================================
\subsection{PotentialCoefficients}\label{general.fileFormat:potentialCoefficients}
The standard \verb|.gfc| format as defined by the ICGEM is used in ASCII the format.
Only the static part is used and temporal variations (e.g. trend) are ignored.
To write additional information and temporal variations use \program{PotentialCoefficients2Icgem}.


%==================================
\subsection{SatelliteModel}\label{general.fileFormat:satelliteModel}
Properties of a satellite to model non-conservative forces (e.g. \configClass{miscAccelerations}{miscAccelerationsType}).
The file may contain surface properties, mass, drag coefficients, and antenna thrust values.

See \program{SatelliteModelCreate} and \program{SinexMetadata2SatelliteModel}.

\begin{verbatim}
<?xml version="1.0" encoding="UTF-8"?>
<groops type="satelliteModel" version="20190429">
   <satelliteCount>1</satelliteCount>
   <satellite>
       <satelliteName>GALILEO-2</satelliteName>
       <mass>7.00000000000000e+02</mass>
       <coefficientDrag>0.00000000000000e+00</coefficientDrag>
       <surfaceCount>15</surfaceCount>
       <surface>
           <type>0</type>
           <normal>
               <x>-1.00000000000000e+00</x>
               <y>0.00000000000000e+00</y>
               <z>0.00000000000000e+00</z>
           </normal>
           <area>4.40000000000000e-01</area>
           <reflexionVisible>0.00000000000000e+00</reflexionVisible>
           <diffusionVisible>7.00000000000000e-02</diffusionVisible>
           <absorptionVisible>9.30000000000000e-01</absorptionVisible>
           <reflexionInfrared>1.00000000000000e-01</reflexionInfrared>
           <diffusionInfrared>1.00000000000000e-01</diffusionInfrared>
           <absorptionInfrared>8.00000000000000e-01</absorptionInfrared>
           <hasThermalReemission>1</hasThermalReemission>
       </surface>
       ...
       <modulCount>2</modulCount>
       <modul>
           <type>1</type>
           <rotationAxis>
               <x>0.00000000000000e+00</x>
               <y>1.00000000000000e+00</y>
               <z>0.00000000000000e+00</z>
           </rotationAxis>
           <normal>
               <x>0.00000000000000e+00</x>
               <y>0.00000000000000e+00</y>
               <z>1.00000000000000e+00</z>
           </normal>
           <surface>
               <count>4</count>
               <cell>11</cell>
               <cell>12</cell>
               <cell>13</cell>
               <cell>14</cell>
           </surface>
       </modul>
       <modul>
           <type>2</type>
           <antennaThrust>
               <x>0.00000000000000e+00</x>
               <y>0.00000000000000e+00</y>
               <z>2.65000000000000e+02</z>
           </antennaThrust>
       </modul>
   </satellite>
</groops>
\end{verbatim}


%==================================
\subsection{StringList}\label{general.fileFormat:stringList}
White space separated list of strings.
Comments are allowed and all the text from the character '\verb|#|' to the end of the line is ignored.
Strings containing white spaces or the '\verb|#|' character must be set in quotes ('\verb|""|').

\begin{verbatim}
# IGSR3 stations
abmf
abpo
ade1
adis
ajac
albh
algo
alic
alrt
amc2
aoml
areq
arev
artu
asc1
\end{verbatim}


%==================================
\subsection{StringTable}\label{general.fileFormat:stringTable}
White space separated table of strings in row and columns.
Additional columns in a row may represent alternatives, e.g. for core stations in a GNSS network.
Comments are allowed and all the text from the character '\verb|#|' to the end of the line is ignored.
Strings containing white spaces or the '\verb|#|' character must be set in quotes  ('\verb|""|').

\begin{verbatim}
# core network with alternative stations
artu mdvj mdvo nril
asc1 sthl
bahr bhr1 yibl nama
chat chti auck
chpi braz ufpr savo
ckis nium
coco xmis dgar dgav
cro1 scub abmf lmmf aoml
daej taej suwn osn1
darw kat1 tow2 alic
dav1 maw1
drao albh will holb nano
fair whit
glps guat
gode godz usno usn3
goug
\end{verbatim}


%==================================
\subsection{TideGeneratingPotential}\label{general.fileFormat:tideGeneratingPotential}

\begin{verbatim}
groops tideGeneratingPotential version=20200123
       7160
# Degree    Dood.    cos                       sin                      name
# ==========================================================================
          3 055.556  0.000000000000000000e+00 -6.569000000000000125e-07 ""
          3 055.635  0.000000000000000000e+00 -2.842000000000000170e-07 ""
          3 055.561  0.000000000000000000e+00 -6.360000000000000040e-08 ""
          2 055.563 -3.122600001000726621e-06  0.000000000000000000e+00 ""
          2 055.565  7.719644799947265879e-02  0.000000000000000000e+00 om1
          3 055.645  0.000000000000000000e+00  2.921429999971616515e-05 ""
          2 055.573  1.975999999999999959e-07  0.000000000000000000e+00 ""
          2 055.575 -7.535264999889109729e-04  0.000000000000000000e+00 om2
\end{verbatim}



%==================================
\subsection{TimeSplinesCovariance}\label{general.fileFormat:timeSplinesCovariance}
Stores covariance information for \file{TimeSplinesGravityField}{timeSplinesGravityField}.
It can be the variances of the potential coefficients or the full covariance matrix for each
temporal nodal point.


%==================================
\subsection{TimeSplinesGravityField}\label{general.fileFormat:timeSplinesGravityField}
Temporal changing gravity field, parametrized as spherical harmonics in the spatial domain and
parametrized as basis splines in the time domain (see~\reference{basis splines}{fundamentals.basisSplines}).
It is evaluated with \configClass{gravityfield:timeSplines}{gravityfieldType:timeSplines}.

See also: \program{Gravityfield2TimeSplines}, \program{PotentialCoefficients2BlockMeanTimeSplines}.


%==================================
\subsection{VariationalEquation}\label{general.fileFormat:variationalEquation}
Arcs with reference orbit and state transition matrices.

The file contains a reference orbit (position and velocity),
the derivatives of the orbit with respect to the satellite state vector for each arc,
transformations (rotations) between the satellite, celestial, and terrestrial frame
and a satellite macro model (see \file{SatelliteModel}{satelliteModel}).

The reference orbit can be extracted with \program{Variational2OrbitAndStarCamera}.

See also: \program{PreprocessingVariationalEquation}.



%==================================
